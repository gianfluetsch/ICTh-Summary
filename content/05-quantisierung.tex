%! Licence = CC BY-NC-SA 4.0

%! Author = gianfluetsch
%! Date = 22. Jan 2022
%! Project = icth_summary

\section{Quantisierung}
\subsubsection{A}
\textbf{Ist es nach einer Quantisierung möglich das originale Analoge Signal zu 100\% wieder herzustellen? (Mit Begründung).}\\
Nein, das analoge Signal wird aus dem digitalen Signal lediglich rekonstriert. Es ist jedoch von geringerer Qualität als das Originalsignal. Diese Differenz zwischen den Signalen wird durch
den Quantisierungsfehler hervorgerufen.\\

\textbf{Welche Grund-Bitrate wird für ein Sprachsignal (digitale Übertragung der Sprache in einem Telefonkanal) benötigt? (Berechnung und Begründung)}\\
Die Sprache hat einen Umfang von0,3-3,4 kHz. Nach inernationalen Normen wird ein Frequenzspektrum von 4kHz angenommen. Daraus folgt die Abtastfrequenz 2*4kHz = 8kHz. Im weiteren sind $2^12$ Quantisierungsstufen zu wählen. Durch die Kompandierung werden diese vor der Übertragung auf 8 bit komprimiert. Dies ergibt eine CW-Länge von 8Bit. Die
Bitrate folgt nun aus der Abtastfrequenz und der CW-Bitlänge 8kHz * 8 Bit = 64kBit/s

\subsubsection{B}
Ein analoges, auf 4 kHz bandbegrenztes Signal s(t) soll in q = 64 Stufen quantisiert werden.

\textbf{Welche Bitrate ist für die digitale Üertragung dieses Signals mindestens erforderlich (theoretischer Grenzwert)?}\\
Es liegt ein bandbegrenztes Signal vor. Gemäss in der Vorlesung vorgestellten Verfahren zur Abtastung muss das Nutzsignal zur fehlerfreien Rückgewinnung mit mindestens der doppelten oberen Grenzfrequenz des Nutzsignals abgetastet werden.\\
Die Wortgrösse ist bei $q=64=2^6$ ist 6 Bit und die Bitrate entsprechend $2*4kHz*6Bit=48kBit/s$\\

\textbf{Wie gross ist die Bitrate, wenn das Signal überabgetastet wird (10\%, 20\%)?}\\
Die Wortgrösse ändert sich nicht durch die Überabtastung natürlich nicht.\\
Die Bitrate ist entsprechend $2*1.1*4kHz*6Bit=52.8kHz resp. 2*1.2*4kHz*6Bit=57.6kHz$\\

\textbf{Ein analoges, amplitudenbegrenztes Signal ($0 \leq  U \leq  1V und 0 \leq  f \leq  100Hz$) wird quantisiert über eine Binärkanal übertragen.}\\
Es können $2*100Hz=200$ Abtastungen pro Sekunde durchgeführt werden. Pro Abtastung stehen $\frac{1200Bit/s}{200Hz}=6Bit$ zur Verfügung. Mit 6 Bit sind $2^6=64$ Stufen möglich, was einer Wertstufe von $\Delta U=\frac{1V}{64}=15.6mV$ entspricht.\\

\columnbreak

\textbf{Wie gross müsste die Übertragungsrate mindestens sein, falls eine Auflösung von $\Delta U = 5mV$ gefordert wäre?}\\
\textit{im fehlerfreien Fall:}\\
Dafür bräcute es $\frac{1V}{5mV}=200$ Stufen, wozu mindestens 8 Bit($2^8=256$) benötigt werden. Die Übertragungsrate wäre dementsprechend $2*100Hz*8=1600Bit/s$.\\

\textit{bei einem fehlerbehafteten Kanal mit der Kanalmatrix:}\\
$P(Y|X)=\begin{bmatrix}
    0.995 & 0.05\\
    0.01 & 0.99\\
\end{bmatrix}$\\

Sie können von einer gleichverteilten binären Quelle ausgehen und die Kanalmatrix als unabhängig von der Taktfrequenz betrachten.\\
Entscheidend für die tatsächliche Übertragung ist die Transinformation. Die Irrelevanz ist:\\

$H(Y|X)=-\sum_{i_1}^2\sum_{k=1}^2p(x_i,y_k)*log_2(p(y_k|x_i))$\\
$=-\sum_{i_1}^2\sum_{k=1}^2p(x_i)*p(y_k|x_i)*log_2(p(y_k|x_i))$\\
$=-0.5*0.995*log_2(0.995)-0.5*0.005*log_2(0.005)$\\
$-0.5*0.99*log_2(0.99)-0.5*0.01*log_2(0.01)$\\
$=0.0631$\\

Bei einer gleichverteilten Quelle ist die Entropie $H(Y) = 1$. Die Transinformation ist damit:\\
$T=H(Y)-H(Y|X)=1-0.0631=0.9369$\\

Um 1600 Bit/s zu übertragen werden damit mindestens $\frac{1600 Bit/s}{0.9369}=1707 Bit/s$ benötigt.
