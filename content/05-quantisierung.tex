%! Licence = CC BY-NC-SA 4.0

%! Author = gianfluetsch
%! Date = 22. Jan 2022
%! Project = icth_summary

\section{Quantisierung}
\subsection{Rinkel}
\subsubsection{Prüfung FS2009}
Ist es nach einer Quantisierung möglich das originale Analoge Signal zu 100\% wieder herzustellen? (Mit Begründung).\\
\textit{Nein, das analoge Signal wird aus dem digitalen Signal lediglich rekonstriert. Es ist jedoch von geringerer Qualität als das Originalsignal. Diese Differenz zwischen den Signalen wird durch
den Quantisierungsfehler hervorgerufen.}\\

Welche Grund-Bitrate wird für ein Sprachsignal (digitale Übertragung der Sprache in einem Telefonkanal) benötigt? (Berechnung und Begründung)\\
\textit{Die Sprache hat einen Umfang von0,3-3,4 kHz. Nach inernationalen Normen wird ein Frequenzspektrum von 4kHz angenommen. Daraus folgt die Abtastfrequenz 2*4kHz = 8kHz. Im weiteren sind $2^12$ Quantisierungsstufen zu wählen. Durch die Kompandierung werden diese vor der Übertragung auf 8 bit komprimiert. Dies ergibt eine CW-Länge von 8Bit. Die
Bitrate folgt nun aus der Abtastfrequenz und der CW-Bitlänge 8kHz * 8 Bit = 64kBit/s}