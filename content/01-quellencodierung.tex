%! Licence = CC BY-NC-SA 4.0

%! Author = gianfluetsch
%! Date = 22. Jan 2022
%! Project = icth_summary

\section{Quellencodierung}

\subsection{Kompression/ Komprimierung}

\subsubsection{Nachprüfung HS2020}
\begin{center}
    \centering
    \begin{tabular}{l | l}
        \bfseries{Zeichen} & \bfseries{Wahrscheinlichkeit p(x)}\\ \hline
        a & 0.4\\ 
        b & 0.3\\
        c & 0.2\\
        d & 0.1\\
    \end{tabular}
\end{center}

\paragraph{Entropie der Quelle}\mbox{}\\
$I(x) = log_{2}(\frac{1}{p(x)})$\\
$H(x) = p(x)*I(x)$\\
$H(x) = 0.4*log_2\frac{1}{0.4}+0.3*log_2\frac{1}{0.3}+0.2*log_2\frac{1}{0.2}+0.1*log_2\frac{1}{0.1}=1.846$

\paragraph{Redundanz der Quelle}\mbox{}\\
$H_0 = log_2(N) = log_2(4) = 2 \rightarrow$ (N = Anzahl Zeichen)\\
$R_0 = H_0 - H(x) = 2-1.846=0.153$

\paragraph{Redundanz des Codes mit Codierung}
\begin{center}
    \centering
    \begin{tabular}{l | l}
        \bfseries{Zeichen} & \bfseries{Codierung}\\ \hline
        a & 001\\ 
        b & 010\\
        c & 011\\
        d & 100\\
    \end{tabular}
\end{center}
$L = p*L \rightarrow p_1*L_1+p_2*L_2...$\\
$R_c = L - H(x) \rightarrow$ L=Anzahl Bits (3)\\
$3-1.846=1.154$

\paragraph{Huffman-Code erstellen (Codierung)}\mbox{}\\
\begin{center}
    \centering
    \begin{tabular}{l | l}
        \bfseries{Zeichen} & \bfseries{Codierung}\\ \hline
        a & 1\\ 
        b & 01\\
        c & 001\\
        d & 000\\
    \end{tabular}
\end{center}

\paragraph{Redundanz Huffman-Code}\mbox{}\\
$L=0.4*1+0.3*2+0.2*3+0.1*3=1.9$\\
$R_c = L-H(x) = 1.9-1.846=0.054$

\paragraph{Codieren sie ''aaabbacd'' mit Huffman-Code}\mbox{}\\
$a\_a\_a\_\_b\_\_b\_\_\_a\_\_c\_\_d$\\
$1\_1\_1\_|\_01\_01\_|\_1\_001\_000 \rightarrow 14 Bits$

\paragraph{Lauflängencodierung der Bitfolge}\mbox{}\\
Orginale Bitfolge: $11101011001000 \rightarrow |w_c| = 14 bit$\\
Code: $3$ $1$ $1$ $1$ $2$ $2$ $1$ $3 \rightarrow$ mit 2 Bits kann Code codiert werden\\
Bitfolge komprimiert: $11$ $01$ $01$ $01$ $10$ $10$ $01$ $11$\\
$\rightarrow |w_c| = 16 bit$

\paragraph{Kompression im Vergleich}\mbox{}\\
Ursprünglich: $8\_Zeichen * 3\_Bit = 24 Bit$\\
Huffman: $14 Bit$\\
$\rightarrow 14Bit/24Bit=0.5833=58\%$

\paragraph{Min. theoretische Redundanz Huffman}\mbox{}\\
???

\paragraph{Wann wird das erreicht?}\mbox{}\\
???

\subsubsection{Prüfung FS2017}
\begin{center}
    \centering
    \begin{tabular}{l | l}
        \bfseries{Zeichen} & \bfseries{Wahrscheinlichkeit p(x)}\\ \hline
        a & 0.5\\ 
        b & 0.25\\
        c & 0.1\\
        d & 0.1\\
        e & 0.05
    \end{tabular}
\end{center}

\paragraph{Entscheidungsgehalt $H_0$ des Codes}\mbox{}\\
$H_0=log_2(N) \rightarrow$ N = Anzahl Zeichen\\
$H_0=log_2(5)=2.322$

\paragraph{Redundanz der Quelle}\mbox{}\\
$H(x) = p(x)*I(x)$\\
$I(x) = log_2(\frac{1}{p(x)})$\\
$H(x) = 0.5*1+0.25*2+0.1*3.322+0.1*3.322+0.05*4.322=1.88$
$R_0 = H_0 - H(x) = 2.322-1.88=0.44$

\paragraph{Mittlere Codewortlänge \&Redundanz des Codes}\mbox{}\\
\begin{center}
    \centering
    \begin{tabular}{p{1cm} | p{1.5cm} | p{1.5cm} | p{1.25cm}}
        \bfseries{Zeichen} & \bfseries{Codierung} & \bfseries{Länge (L)} & \bfseries{Mittlere Länge}\\ \hline
        a & 1 & 1 & 0.5\\ 
        b & 01 & 2 & 0.5\\
        c & 0100 & 4 & 0.4\\
        d & 1000 & 4 & 0.4\\
        e & 11000 & 4 & 0.25
    \end{tabular}
\end{center}
$L = p*L \rightarrow p_1*L_1+p_2*L_2...$\\
$L=0.5+0.5+0.4+0.4+0.25=2.05$\\
$R_c = L - H(x)$\\
$2.05-1.88=0.17$

\paragraph{Optimierung Codierung nach Huffman}\mbox{}\\
Wie viel Prozent kann Redundanz in Code mit Huffman gegenüber ursprünglichem Code reduziert werden?\\
\begin{center}
    \vspace{-8pt}
    \includegraphics[width=.5\linewidth]{./01-quellencodierung/optimierung_codierung}
    \vspace{-8pt}
\end{center}

\begin{center}
    \centering
    \begin{tabular}{l | l}
        \bfseries{Zeichen} & \bfseries{Huffman}\\ \hline
        a & 1\\ 
        b & 01\\
        c & 001\\
        d & 0001\\
        e & 0000\\
    \end{tabular}
\end{center}

Daraus folgt für die Redundanz nach Huffman = 0.02. 
Das sind 11.8\% und entspricht einer Redundanzreduktion um 88.2\%.

\subsubsection{Prüfung HS2016}
Der Informationsgehalt einer Bildvorlage soll über einen Binärkanal mit der Kanalkapaziät von 40 kBit/sec übertragen werden. Dabei wird das Bild durch Abtastung in 105 Bildpunkte zerlegt. Ein Bild-punkt kann in 8 Helligkeitswerte H zerlegt werden.

\paragraph{Welche Übertragungszeit ergibt sich, wenn alle Helligkeitswerte gleich codiert werden?}\mbox{}\\
$L_D 8=3$ Bit pro Bildpunkt\\
$\rightarrow TÜ = 3*10^5:40kBit/sec=7.5sec$\\

Nähere Untersuchungen der Helligkeitswerte ergeben für die Helligkeitswerte H1 bis H8 folgende Ver-teilung ihrer Auftrittswahrscheinlichkeit:
\begin{center}
    \centering
    \begin{tabular}{l | l}
        \bfseries{H} & \bfseries{P}\\ \hline
        H1 & 0.4\\ 
        H2 & 0.35\\
        H7 & 0.1\\
        H4 & 0.0625\\
        H3 & 0.035\\
        H5 & 0.03225\\
        H6 & 0.01425\\
        H8 & 0.006
    \end{tabular}
\end{center}

\paragraph{Um wie viel Prozent können sie die Übertragungsgeschwindigkeit eines Bildes verkürzen, wenn sie die erfassten Bildpunkte vor dem Versenden nach Huffmann codieren?}\mbox{}\\
\begin{center}
    \centering
    \begin{tabular}{p{1cm} | p{1cm} | p{1cm} | p{1cm} | p{1cm}}
        \bfseries{H} & \bfseries{P} & \bfseries{Möglicher Code} & \bfseries{Anzahl Bits} & \bfseries{Länge}\\ \hline
        H1 & 0.4 & 0 & 1 & 0.4\\ 
        H2 & 0.35 & 10 & 2 & 0.7\\
        H7 & 0.1 & 110 & 3 & 0.3\\
        H4 & 0.0625 & 1110 & 4 & 0.25\\
        H3 & 0.035 & 11110 & 5 & 0.175\\
        H5 & 0.03225 & 111110 & 6 & 0.1935\\
        H6 & 0.01425 & 1111110 & 7 & 0.09975\\
        H8 & 0.006 & 1111111 & 7 & 0.042\\
        Total & & & & 2.16025
    \end{tabular}
\end{center}

$TÜ=2.16*10^5:40kBit/sec = 5.4sec$\\
$7.5=100\%$\\
$5.4 = 5.4/7.5*100\% = 72\%$\\
Daraus folgt eine Verbesserung um 28\%.

\paragraph{Für den Kanal konnte die folgende Kanalmatrix P(Y|X) ermittelt werden.}\mbox{}\\
$P(X|Y) = \begin{matrix}
    0.9 & 0.1\\
    0.1 & 0.9\\
\end{matrix}$

Wie lange dauert es dann mindestens, bis alle Daten fehlerfrei übertragen sind? Gehen Sie davon aus das „0“ und  „1“ gleich wahrscheinlich sind. \\
$TÜ = 2.16*10^5:(1000*40Bit)=5.4sec$


\subsection{Verschlüsselung}
\subsubsection{Nachprüfung HS2020}
Mit RSA-verschlüsselte Nachricht c=8 abgefangen und kennen öffentlichen Schlüssel des Empfängers (e=13, n=323).
Aus sicherer Quelle wissen Sie, dass eine der beiden Primzahlen 17 ist. Entschlüsseln Sie die Nachricht!\\
Gegeben: $c=8, e=13, n=323, p=17$\\
$solve(323=17*x,x) \rightarrow q=19$\\
$\phi(n) = (p-1)*(q-1) = 16*18=288$\\
$d=\frac{2*\phi(n)+1}{e}=\frac{2*288+1}{13}$
???

\subsubsection{Prüfung HS2020}
Mit RSA-verschlüsselte Nachricht c=4 abgefangen und kennen öffentlichen Schlüssel des Empfängers (e=13, n=143).
Beide Primzahlen (p \& q) sind grösser als 10. Entschlüsseln Sie die Nachricht!\\
Gegeben: $c=8, e=13, n=143$\\
$p=11, q=13 \rightarrow 11*13 = 143$\\
$\phi(n) = (p-1)*(q-1) = 10*12=120$\\
$d=\frac{2*\phi(n)+1}{e}=\frac{2*288+1}{13}$
???

\subsubsection{Prüfung FS2017}
Es wird ein asymmetrisches Verfahren (RSA) verwendet. Wie viele verschiedene Schlüsselpaare müssen erzeugt werden, wenn in einer Gruppe von 25 Teilnehmern. jeder mit jedem vertraulich kommunizieren möchte?\\
\textit{25 Schlüsselpaare}\\

Wie viele Schlüssel muss jeder Teilnehmer speichern, wenn sie davon ausgehen, dass es eine vertrauenswürdige „Schlüsselbank“ gibt, bei der die öffentlichen Schlüssel abgefragt werden können?\\
\textit{Bei einer öffentlichen Schlüsselbank. Nur seinen Private Key (und ggf. noch seinen öffentlichen).}\\

Teilnehmer a möchte an Teilnehmer b eine verschlüsselte und signierte Nachricht versenden. Welche Schlüssel verwendet
\begin{itemize}
    \item User a: \textit{public Key von b und private Key von a}
    \item User b: \textit{public Key von a und private Key von b}\\
\end{itemize}

Kommt es bei der Anwendung durch die Teilnehmer a und b dabei auf die Reihenfolge der Schlüssel an?\\
\textit{Nein}\\

\paragraph{Gegeben seien die beiden Primzahlen 11 und 5}\mbox{}\\
Bestimmen sie die Zahl \textit{n} und die Zahl $\phi(n)$\\
$n=11*5=55$\\
$\phi(n) = 10*4=40$

\paragraph{Zur Verschlüsselung wird der öffentliche Schlüssel e  = 17 verwendet}\mbox{}\\
Ermitteln sie den privaten Schlüssel d zum Entschlüsseln einer Nachricht.\\
Es muss gehlten: $e*dmod\phi(n)=1$\\

\begin{center}
    \vspace{-8pt}
    \includegraphics[width=.5\linewidth]{./01-quellencodierung/fs2017_4}
    \vspace{-8pt}
\end{center}

Daraus folgt, 33 ist die Inverse zu 17 bzgl. der mod40 Rechnung. D.h. der Schlüssel zum Entschlüsseln ist 33.

\subsubsection{FS2016}
91 Teilnehmer wollen paarweise vertraulich Informationen austauschen.\\

In einem ersten Ansatz wird ein symmetrisches Verfahren gewählt. 
Wie viele verschiedene Schlüssel müssen erzeugt werden, wenn jeder mit jedem vertraulich kommunizieren möchte?\\
\textit{91/2*90=4095}\\

Wie viele Schlüssel muss jeder Teilnehmer speichern?\\
\textit{N=90}\\



